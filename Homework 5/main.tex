\documentclass[11pt]{article}
\usepackage[utf8]{inputenc}
\usepackage[margin=1in]{geometry}
\usepackage{siunitx,pgfplots,fancyhdr,enumitem,tikz,xparse,listings,graphicx,float}
\usepackage[T1]{fontenc}

\usepackage[colorlinks=true]{hyperref}

\lstset{language=Matlab}

%%Needed to properly display graphs
\pgfplotsset{width=10cm,compat=1.9}

\pagestyle{fancy}
\fancyhf{}
\lhead{Steven Glasford}
\chead{Homework 5}
\rhead{Page \thepage}

\title{Homework 5}
\author{Steven Glasford}
\date{\parbox{\linewidth}{\centering%
    %%Adds the last compiled date
    \today\endgraf\medskip
    Numerical Analysis\endgraf\medskip
    MATH-373}}

%%Creates a new symbol for plus and minus together
\newcommand{\rpm}{\sbox0{$1$}\sbox2{$\scriptstyle\pm$}
  \raise\dimexpr(\ht0-\ht2)/2\relax\box2 }

%%Creates a nice format for displaying the steps taken  
\newlist{steps}{enumerate}{1}
\setlist[steps, 1]{label = Step \arabic*:}

\ExplSyntaxOn
%%new command to round numbers
\newcommand*{\prlen}[1]{%
   % round to 1 digit:
    \pgfmathparse{round(10)/10.0}%
    %\pgfkeys{/pgf/number format/precision=1}
    %\pgfmathresult
    \pgfmathprintnumber[fixed, precision=2]{\pgfmathresult}
}
\ExplSyntaxOff

% Set the listings programming language
\lstset{language=Matlab}

\begin{document}

%% adds the title to the document
\maketitle
\pagebreak
%\tableofcontents

\section{Problem Statement}

    The force on a sailboat mast can be represented by the following function: 
    $$f(z) = 200\left(\frac{z}{5+z}\right)e^{-2z/H}$$ 
    where $z=$ the elevation above the deck and $H=$ the height of the mast. The total force $F$ exerted on the mast can be determined by integrating this function over the height of the mast: 
    $$F = \int_{0}^{H} f(z) dz$$
    The line of action can also be determined by integration: 
    $$d = \frac{\int_{0}^{H} z f(z) dz}{\int_{0}^{H} f(z) dz}$$
    Solve using the following methods where $H=30$ feet:
    
	\begin{enumerate}
	
	\item\label{part1}  Use Romberg integration to find both $F$ and $d$.
	
	\item\label{part2}  Use Gauss Quadrature with four points to find both $F$ and $d$.
	
	\item\label{part3}  Use Gauss Quadrature with five points to find both $F$ and $d$.
	
	\item\label{part4}  Use Adaptive Quadrature to find both $F$ and $d$. 
	
	\end{enumerate}
    Assume that units for $H$ are in feet.  This solution must be typeset using \LaTeX\ and a hard copy should be submitted in class.  Be sure to follow the template for homework found elsewhere.


\section{Solution}

\begin{table}[H]
    \label{tab:solution}
    \caption{Solutions for $F$, and $d$ as computed by various numerical methods}
\begin{center}
\begin{tabular}{|c||c||c|}
     \hline
     Problem&$F$&$d$  \\
     \hline \hline
     \begin{flushleft} \ref{part1}: Romberg\end{flushleft} & $\num{1.4806e3}$ & 13.0537 \\
     \hline
     \begin{flushleft} \ref{part2}: Gauss Quadrature (4 points) \end{flushleft}& $\num{1.4868e3}$ & 12.9778 \\
     \hline
     \begin{flushleft} \ref{part3}: Gauss Quadrature (5 points)\end{flushleft} & $\num{1.4819e3}$ & 13.0379\\
     \hline
     \begin{flushleft} \ref{part4}: Adaptive Quadrature\end{flushleft} & $\num{1.4806e3}$ & 13.0537\\
     \hline
\end{tabular}
\end{center}
\end{table}


\medskip

The code used to calculate Romberg can be found in Code \ref{code:romberg}.\\
The code used to calculate Gauss Quadrature can be found in Code \ref{code:gauss}.\\
The code used to calculate Adaptive Quadrature can be found in Code \ref{code:step} and \ref{code:adapt}.



\section{Command-line Usage}
One program was written to produce the values found in Table \ref{tab:solution}, the code for this program can be found in Code \ref{code:hw5} and is ran as follows: 

\begin{verbatim}
>> [RombergF,Rombergd,Gauss4F,Gauss4d,Gauss5F,Gauss5d,AdaptF,Adaptd] = hw5()
\end{verbatim}

\section{Code}

\lstinputlisting[caption={MATLAB Code \texttt{hw5.m}}, label = {code:hw5},frame=tb]{hw5.m}

\lstinputlisting[caption={MATLAB Code \texttt{romberg.m}}, label = {code:romberg},frame=tb]{romberg.m}

\lstinputlisting[caption={MATLAB Code \texttt{gauss\_quad.m}}, label = {code:gauss},frame=tb]{gauss_quad.m}

\lstinputlisting[caption={MATLAB Code \texttt{quad\_step.m}}, label = {code:step},frame=tb]{quad_step.m}

\lstinputlisting[caption={MATLAB Code \texttt{quad\_adapt.m}}, label = {code:adapt},frame=tb]{quad_adapt.m}

\end{document}