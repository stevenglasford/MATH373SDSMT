\documentclass[11pt]{article}
\usepackage[utf8]{inputenc}
\usepackage[margin=1in]{geometry}
\usepackage{siunitx,pgfplots,fancyhdr,enumitem,tikz,xparse,listings,graphicx}
\usepackage[T1]{fontenc}

\usepackage[colorlinks=true]{hyperref}

\lstset{language=Matlab}

%%Needed to properly display graphs
\pgfplotsset{width=10cm,compat=1.9}

\pagestyle{fancy}
\fancyhf{}
\lhead{Steven Glasford}
\chead{Homework 3}
\rhead{Page \thepage}

\title{Homework 3}
\author{Steven Glasford}
\date{\parbox{\linewidth}{\centering%
    %%Adds the last compiled date
    \today\endgraf\medskip
    Numerical Analysis\endgraf\medskip
    MATH-373}}

%%Creates a new symbol for plus and minus together
\newcommand{\rpm}{\sbox0{$1$}\sbox2{$\scriptstyle\pm$}
  \raise\dimexpr(\ht0-\ht2)/2\relax\box2 }

%%Creates a nice format for displaying the steps taken  
\newlist{steps}{enumerate}{1}
\setlist[steps, 1]{label = Step \arabic*:}

\ExplSyntaxOn
%%new command to round numbers
\newcommand*{\prlen}[1]{%
   % round to 1 digit:
    \pgfmathparse{round(10)/10.0}%
    %\pgfkeys{/pgf/number format/precision=1}
    %\pgfmathresult
    \pgfmathprintnumber[fixed, precision=2]{\pgfmathresult}
}
\ExplSyntaxOff

% Set the listings programming language
\lstset{language=Matlab}

\begin{document}

%% adds the title to the document
\maketitle
\pagebreak
%\tableofcontents

\section{Problem Statement}
The equation for the current, $i$, in a circuit with a resistor, capacitor, and inductor is given by 
$$\frac{i}{i_0} = e^{-Rt/(2L)} \cos{\left( t \sqrt{\frac{1}{LC} - \left(\frac{R}{2L} \right)^2}\right)},$$ where $i_0$ 
is the initial current in the circuit, $R$ is the resistance in the circuit, $L$ is the inductance in the circuit, $C$ is the capacitance in the circuit, and $t$ is the duration of the discharge.  Overall, the charge stored in the capacitor drives the current through the circuit, while the resistor and inductor dissipate the current.

Determine the half-lie of the current (i.e., the time until $i/i_0 = 0.5$) if $R = 100$ \si{\ohm}, $L = 5$ \si{\henry}, and $C = 10^{-4}$ \si{\farad} using IQI and the MATLAB command \texttt{fzero}.  Report the outputs for the two methods in a properly formatted table.  You do not need to include a plot for this report, but if you do not include a plot, then make sure you address how you determined an interval or initial guesses for your methods.  This solution must be typeset using \LaTeX\ and a hard copy should be submitted in class.  Be sure to follow the template for homework on the course website.

\section{Solution}

Replacing all of the variables in the function from above with the constants given and solving the equation for zero results in the following function and graph \ref{fig:function}:


$$0 = e^{-100t/(2\cdot 5)} \cos{\left( t \sqrt{\frac{1}{5\cdot 10^{-4}} - \left(\frac{100}{2\cdot 5} \right)^2}\right)} - 0.5,$$

or simplified:

$$0 = e^{-10t} \cos{\left(  10\cdot\sqrt{19}t\right)} - 0.5$$

\bigskip


\begin{center}
\begin{tikzpicture}
\begin{axis}[
    axis lines = left,
    xlabel = $x$,
    ylabel = {$f(x)$},
    scaled ticks = false,
    tick label style = {/pgf/number format/fixed},
]
%Below the red parabola is defined
\addplot [
    domain=0:.04, 
    samples=100, 
    color=red,
]
{exp(-10*x)*cos(deg(x*sqrt(1900)))-.5};
\addlegendentry{$e^{-10t} \cos{\left(  10\cdot\sqrt{19}t\right)} - 0.5$}

\draw[ultra thin] (axis cs:\pgfkeysvalueof{/pgfplots/xmin},0) -- (axis cs:\pgfkeysvalueof{/pgfplots/xmax},0);

\end{axis}
\end{tikzpicture}
\end{center}



We can see that the function has a root somewhere between 0 and 0.04, so we can set the bounds of the \texttt{fzero} function to $[0,0.04]$, and it be logical to set the guesses for the \texttt{IQI} function to $[0,0.02,0.04]$.

\begin{center}
\verb|func = @(t) exp(-10*x)*cos(x*sqrt(1900));|
\end{center}

\begin{center}
    \begin{tabular}{c|c|c}
        & \texttt{IQI(func,0,.02,.04)} & \texttt{fzero(func,[0,.04])}\\
        \hline
        \texttt{x\_root} & 0.0208 & 0.0208 \\
        \texttt{func\_val} & $\num{7.2276e-14}$ \\
        \texttt{error\_approx} & $\num{4.7841e-05}$ \\
        \texttt{num\_iterations} & 4
        
    \end{tabular}
\end{center} 

\section{Command-line Usage}
The graph for this document was produced natively inside latex. And the data for the table was produced in a MATLAB file called \texttt{hw.m}, the code for which can be found in Listing \ref{code:hw3}. The code for IQI can be found in Listing \ref{code:iqi}.

\begin{verbatim}
>> hw3()
\end{verbatim}

\section{Code}

\lstinputlisting[caption={MATLAB Code to Solve the Problem}, label = {code:hw3}, frame=tb]{h3.m}

\lstinputlisting[caption={MATLAB Code for IQI}, label = {code:iqi},frame=tb]{IQI.m}

\end{document}